\documentclass[12pt]{article}

\usepackage[spanish]{babel}
\decimalpoint % to render points as points for decimal separators

\usepackage[utf8]{inputenc}

\usepackage[left=1cm, right=1cm, top=1cm]{geometry} 

\usepackage{amsmath}
\usepackage{amssymb}
%\usepackage{upgreek}
\usepackage{xcolor}
\usepackage{eurosym}
\usepackage{graphicx}
%%%%%%%%%%%%%%%%%%%%%%%%%%%%%%%%%%%%%%%%%%%%%%%%

\newcommand{\R}[1][]{\mathbb{R}^{#1}}
\newcommand{\N}[1][]{\mathbb{N}^{#1}}
\newcommand{\solution}[1]{\text{\fbox{$#1$}}}

\DeclareMathOperator\arctanh{arctanh}

\newcommand{\abs}[1]{\left|#1\right|}

\newcommand{\FVCP}[5]{x(t)=e^{#1}\int_{#4}^{t}e^{#2}#3ds+#5e^{#1}}
\newcommand{\FVC}{x(t)=e^{\int_{t_0}^{t}a(s)ds}\int_{t_0}^te^{-\int_{t_0}^{s}a(z)dz}b(s)ds+x_0e^{\int_{t_0}^{t}a(s)ds}, \quad x(t_0)=x_0}


\newcommand{\tick}{\textbf{\color{green}{ (\checkmark) }}}
\newcommand{\warning}{\textbf{\color{red}{ {\fontencoding{U}\fontfamily{futs}\selectfont\char 66\relax} }}}

\newcommand{\U}{\mathcal{U}}

\newcommand{\qued}{\hfill$\blacksquare$}
\newtheorem{theorem}{Teorema:}
\newtheorem{proof}{Demostración:}

%%%%%%%%%%%%%%%%%%%%%%%%%%%%%%%%%%%%%%%%%%%%%%%%

\newenvironment{aclaration}    
{
\begin{center}
\begin{tabular}{|p{0.9\textwidth}|}
\hline \\ \warning
}{
\\\hline
\end{tabular} 
\end{center}
}

%%%%%%%%%%%%%%%%%%%%%%%%%%%%%%%%%%%%%%%%%%%%%%%%

\begin{document}

%%%%%%%%%%%%%%%%%%%%%%%%%%%%%%%%%%%%%%%%%%%%%%%%

\author{Antonio Gámiz Delgado}
\title{Ejercicio Propuesto II}
\maketitle

%%%%%%%%%%%%%%%%%%%%%%%%%%%%%%%%%%%%%%%%%%%%%%%%
\begin{theorem}
Sea $\Omega$ un dominio estrellado y $P,Q\in C^1(\Omega)$ que cumple la condición de exactitud $\left(\displaystyle \frac{\partial P}{\partial y}= \frac{\partial Q}{\partial x}\right)$. Entonces:
\[
\exists \U\in C^1(\Omega): \quad \frac{\partial\U}{\partial x}=P \quad \frac{\partial\U}{\partial y}=Q.
\]
\end{theorem}


\begin{proof}
Como $\Omega$ es dominio estrellado, entonces $\exists z_*\in\Omega:\enskip z_*=(\alpha, \beta)$ con $\alpha,\beta \in \R$. Por definición de dominio estrellado tenemos que:
\[
[(x,y),(\alpha, \beta)]=\{(\lambda(x-\alpha)+\alpha,\lambda(y-\beta)+\beta),\lambda\in[0,1]\}
\]

Podemos definir la función $\U: \Omega	\longrightarrow \R$ como:
\[
\U(\lambda(x-\alpha)+\alpha,\lambda(y-\beta)+\beta)=x\int_0^1P(\lambda(x-\alpha)+\alpha,\lambda(y-\beta)+\beta)d\lambda+y\int_0^1Q(\lambda(x-\alpha)+\alpha,\lambda(y-\beta)+\beta)d\lambda
\]

Calculamos ahora $\displaystyle\frac{\partial \U}{\partial x}$ (x es un parámetro):
\begin{align*}
\frac{\partial \U}{\partial x}=& \int_0^1 P(\lambda(x-\alpha)+\alpha,\lambda(y-\beta)+\beta)d\lambda+x\int_0^1\lambda\frac{\partial P}{\partial x}(\lambda(x-\alpha)+\alpha,\lambda(y-\beta)+\beta)d\lambda+ \\
& +y\int_0^1\lambda\frac{\partial Q}{\partial x}(\lambda(x-\alpha)+\alpha,\lambda(y-\beta)+\beta)d\lambda
\end{align*}
Aplicando la condición de exactitud:
\begin{align*}
\frac{\partial \U}{\partial x}=& \int_0^1 P(\lambda(x-\alpha)+\alpha,\lambda(y-\beta)+\beta)d\lambda+x\int_0^1\lambda\frac{\partial P}{\partial x}(\lambda(x-\alpha)+\alpha,\lambda(y-\beta)+\beta)d\lambda+ \\
& +y\int_0^1\lambda\frac{\partial P}{\partial y}(\lambda(x-\alpha)+\alpha,\lambda(y-\beta)+\beta)d\lambda
\end{align*}
Sacando $\lambda$ factor común:
\begin{align*}
\frac{\partial \U}{\partial x}=& \int_0^1 P(\lambda(x-\alpha)+\alpha,\lambda(y-\beta)+\beta)d\lambda+\\
& +\int_0^1\lambda \left(x\frac{\partial P}{\partial x}(\lambda(x-\alpha)+\alpha,\lambda(y-\beta)+\beta)+y\frac{\partial P}{\partial y}(\lambda(x-\alpha)+\alpha,\lambda(y-\beta)+\beta)\right)d\lambda
\end{align*}

Usando que lo del interior de los paréntesis del segundo sumando en la expresión anterior es igual a $\displaystyle\frac{d}{d\lambda}\left(P(x-\alpha)+\alpha,\lambda(y-\beta)+\beta)\right)$, tenemos:

\[
\frac{\partial \U}{\partial x}=\int_0^1 P(\lambda(x-\alpha)+\alpha,\lambda(y-\beta)+\beta)d\lambda+\int_0^1\lambda \left[\displaystyle\frac{d}{d\lambda}\left(P(\lambda(x-\alpha)+\alpha,\lambda(y-\beta)+\beta)\right)\right]d\lambda	
\]

Integrando por partes el segundo sumando:

\[
\left \{ 
\begin{array}{llll}
u & = \lambda &\longrightarrow & du=dv\\
dv & = \frac{d}{d\lambda}P(\lambda(x-\alpha)+\alpha,\lambda(y-\beta)+\beta) &\longrightarrow & v=P(\lambda(x-\alpha2)+\alpha,\lambda(y-\beta)+\beta)
\end{array}
\right\}
\]
\[
[\lambda P(\lambda(x-\alpha)+\alpha,\lambda(y-\beta)+\beta)]^1_0-\int_0^1P(\lambda(x-\alpha)+\alpha,\lambda(y-\beta)+\beta)d\lambda=
\]
\[
=P(x,y)-\int_0^1P(\lambda(x-\alpha)+\alpha,\lambda(y-\beta)+\beta)d\lambda
\]

Y sustituyendo en la fórmula anterior, nos queda:
\[
\frac{\partial \U}{\partial x} = P(x,y)
\]

Realizando el mismo procedimiento, obtenemos $\displaystyle\frac{\partial \U}{\partial y}=Q(x,y)$. 

Falta ver si $\U$ es de clase $C^2(\Omega)$, que lo es al ser $\U$  y sus parciales $\frac{\partial\U}{\partial x}=P$ y $\frac{\partial\U}{\partial y}=Q$ de clase $C^1(\Omega)$.  

\qued

\end{proof}






\end{document}