\documentclass[12pt]{article}

\usepackage[spanish]{babel}
\decimalpoint % to render points as points for decimal separators

\usepackage[utf8]{inputenc}

\usepackage[left=1cm, right=1cm, top=1cm]{geometry} 

\usepackage{amsmath}
\usepackage{amssymb}
%\usepackage{upgreek}
\usepackage{xcolor}
\usepackage{eurosym}
\usepackage{graphicx}
%%%%%%%%%%%%%%%%%%%%%%%%%%%%%%%%%%%%%%%%%%%%%%%%

\newcommand{\R}[1][]{\mathbb{R}^{#1}}
\newcommand{\N}[1][]{\mathbb{N}^{#1}}
\newcommand{\solution}[1]{\text{\fbox{$#1$}}}

\DeclareMathOperator\arctanh{arctanh}

\newcommand{\abs}[1]{\left|#1\right|}

\newcommand{\FVCP}[5]{x(t)=e^{#1}\int_{#4}^{t}e^{#2}#3ds+#5e^{#1}}
\newcommand{\FVC}{x(t)=e^{\int_{t_0}^{t}a(s)ds}\int_{t_0}^te^{-\int_{t_0}^{s}a(z)dz}b(s)ds+x_0e^{\int_{t_0}^{t}a(s)ds}, \quad x(t_0)=x_0}


\newcommand{\tick}{\textbf{\color{green}{ (\checkmark) }}}
\newcommand{\warning}{\textbf{\color{red}{ {\fontencoding{U}\fontfamily{futs}\selectfont\char 66\relax} }}}

\newcommand{\U}{\mathcal{U}}

\newcommand{\qued}{\hfill$\blacksquare$}
\newtheorem{theorem}{Teorema:}
\newtheorem{proof}{Demostración:}

%%%%%%%%%%%%%%%%%%%%%%%%%%%%%%%%%%%%%%%%%%%%%%%%

\newenvironment{aclaration}    
{
\begin{center}
\begin{tabular}{|p{0.9\textwidth}|}
\hline \\ \warning
}{
\\\hline
\end{tabular} 
\end{center}
}

%%%%%%%%%%%%%%%%%%%%%%%%%%%%%%%%%%%%%%%%%%%%%%%%

\begin{document}

%%%%%%%%%%%%%%%%%%%%%%%%%%%%%%%%%%%%%%%%%%%%%%%%

\author{Antonio Gámiz Delgado}
\title{Ejercicio Propuesto I}
\maketitle

\begin{enumerate}
\item Resuelve la ecuación $\displaystyle x'=\alpha x\left(1-\frac{x}{L(t)}\right)$. (Suponiendo $L(t)\neq 0 \enskip \forall t$).

Como la ecuación no es lineal, hagamos el siguiente cambio para convertirla en una lineal:

\[
\varphi : \left \{ 
\begin{array}{ll}
s =& t \\
y =& x^{-1} \Longrightarrow y'=\displaystyle-\frac{x'}{x^2} \Longrightarrow x'=-x^2y'=-\frac{y'}{y^2}
\end{array}
\right.
\]

Haciendo el cambio:
\[
-\frac{y'}{y^2}=\frac{\alpha}{y}\left( 1- \frac{1}{yL(t)} \right) \Longrightarrow y'=-\alpha y\left( 1- \frac{1}{yL(t)} \right)=-\alpha y + \frac{\alpha}{L(t)} \Longrightarrow y'=-\alpha y + \frac{\alpha}{L(t)}
\]

Ya tenemos una ecuación lineal de orden 1, por lo que podemos aplicar la Fórmula de Variación de Constantes ($y(t_0)=K$):

\[
y(t)=e^{-\alpha t}\int_{t_0}^t \frac{\alpha}{L(s)} e^{\alpha s}ds+Ke^{-\alpha t}, \enskip K\in\R.
\]

Por lo que deshaciendo el cambio nos queda:

\[
x(t) = \frac{e^{\displaystyle\alpha t}}{K+\displaystyle\int_{t_0}^t\frac{\alpha}{L(s)} e^{\alpha s}, ds}, \enskip K\in\R.
\]

\end{enumerate}

\end{document}