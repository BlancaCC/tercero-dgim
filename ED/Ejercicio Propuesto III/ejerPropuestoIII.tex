\documentclass[12pt]{article}

\usepackage[spanish]{babel}
\decimalpoint % to render points as points for decimal separators

\usepackage[utf8]{inputenc}

\usepackage[left=2cm, right=2cm, top=1cm]{geometry} 

\usepackage{amsmath}
\usepackage{amssymb}
%\usepackage{upgreek}
\usepackage{xcolor}
\usepackage{eurosym}
\usepackage{graphicx}
\usepackage{amsthm}
%%%%%%%%%%%%%%%%%%%%%%%%%%%%%%%%%%%%%%%%%%%%%%%%

\newcommand{\R}[1][]{\mathbb{R}^{#1}}
\newcommand{\N}[1][]{\mathbb{N}^{#1}}
\newcommand{\solution}[1]{\text{\fbox{$#1$}}}

\DeclareMathOperator\arctanh{arctanh}

\newcommand{\abs}[1]{\left|#1\right|}

\newcommand{\FVCP}[5]{x(t)=e^{#1}\int_{#4}^{t}e^{#2}#3ds+#5e^{#1}}
\newcommand{\FVC}{x(t)=e^{\int_{t_0}^{t}a(s)ds}\int_{t_0}^te^{-\int_{t_0}^{s}a(z)dz}b(s)ds+x_0e^{\int_{t_0}^{t}a(s)ds}, \quad x(t_0)=x_0}


\newcommand{\tick}{\textbf{\color{green}{ (\checkmark) }}}
\newcommand{\warning}{\textbf{\color{red}{ {\fontencoding{U}\fontfamily{futs}\selectfont\char 66\relax} }}}

\newcommand{\U}{\mathcal{U}}

\newcommand{\qued}{\hfill$\blacksquare$}
\newtheorem*{theorem*}{Ejercicio}

\newtheorem*{proof*}{Solución}
\newcommand{\p}[2]{\frac{\partial#1}{\partial#2}}

%%%%%%%%%%%%%%%%%%%%%%%%%%%%%%%%%%%%%%%%%%%%%%%%

\newenvironment{aclaration}    
{
\begin{center}
\begin{tabular}{|p{0.9\textwidth}|}
\hline \\ \warning
}{
\\\hline
\end{tabular} 
\end{center}
}

%%%%%%%%%%%%%%%%%%%%%%%%%%%%%%%%%%%%%%%%%%%%%%%%

\begin{document}

%%%%%%%%%%%%%%%%%%%%%%%%%%%%%%%%%%%%%%%%%%%%%%%%

\author{Antonio Gámiz Delgado}
\title{Ejercicio Propuesto III}
\maketitle

%%%%%%%%%%%%%%%%%%%%%%%%%%%%%%%%%%%%%%%%%%%%%%%%

\begin{theorem*}
Se define $\Omega=\{(x,y)\in\R[2]:x>0, \enskip \frac{1}{4}<x^2+y^2<4\}$. Encuentra un potencial para las funcions $P$ y $Q$.
\[
P(x,y)=-\frac{y}{x^2+y^2} \quad Q(x,y)=\frac{x}{x^2+y^2}
\]
\end{theorem*}

\begin{proof*}
Sabemos que $P$ y $Q$ cumplen la condición de exactitud:
\[
\p{P}{y}=\frac{x^2-y^2}{(x^2+y^2)^2}=\p{Q}{x}
\]

Queremos buscar una función $\U$ tal que:
\[
\p{\U}{x}(x,y)+\p{\U}{y}(x,y)y'=0 \enskip \text{ con } \p{\U}{x}=P \quad \p{\U}{y}=Q
\]

Entonces:
\[
\U(x,y)=\int Pdx=-\int \frac{y}{x^+y^2}dx=-\int \frac{1}{y}\frac{1}{1+(\frac{x}{y})^2}dx=-\arctan\left(\frac{x}{y}\right)+c(y) 
\]
Usando ahora que $\displaystyle\p{\U}{y}=Q$:
\[
\p{\U}{y}=-\frac{1}{1+(\frac{x}{y})^2}\left(-\frac{x}{y^2}\right)+c'(y)=\frac{x}{x^2+y^2}+c'(y)=\frac{x}{x^2+y^2}=Q \Longrightarrow c'(y)=0 \Longrightarrow c(y)=k
\]
Por lo que la función potencial $\U$ que buscamos es:
\[
\U(x,y)=-\arctan\left(\frac{x}{y}\right)+k, \enskip k\in\R
\]
\end{proof*}

\newpage

\begin{theorem*}
Dada la ecuación lineal
\[
a(t)x+b(t)-x'=0
\]
que admite un factor integrante dependiente solamente de $t$: $\mu(t,x)=e^{-A(t)}$ con $A(t)$ primitiva de $a(t)$. Recupera la fórmula de variación de constantes buscando el potencial asociado.
\end{theorem*}
\begin{proof*}
Tenemos que $P(t,x)=a(t)x+b(t)$ y $Q(t,x)=-1$.

Multiplicamos la ecuación original por el factor integrante:
\[
e^{-A(t)}a(t)x+e^{-A(t)}b(t)-e^{-A(t)}x'=0
\]

Queremos buscar una función $\U$ tal que:
\[
\p{\U}{t}(t,x)+\p{\U}{x}(t,x)x'=0 \enskip \text{ con } \p{\U}{t}=P \quad \p{\U}{x}=Q
\]

Entonces:
\[
\U(t,x)=\int Pdx= \int\left(e^{-A(t)}a(t)x+e^{-A(t)}b(t)\right)dt=-xe^{-A(t)}+\int e^{-A(t)}b(t)dt+c(x)
\]
Usando ahora que $\displaystyle\p{\U}{x}=Q=-e^{-A(t)}$:
\[
\p{\U}{x}=-e^{-A(t)}+c'(x)=-e^{-A(t)}\Longrightarrow c(x)=k
\]
Por lo que la función potencial $\U$ que buscamos es:
\[
U(t,x)=-xe^{-A(t)}+\int e^{-A(s)}b(s)ds+k=0
\]

Despejando $x$:
\[
x(t)=e^{A(t)}\left(\int^t_k e^{-A(s)}b(s)ds+k\right) \text{ F.V.C. }
\]
\end{proof*}
\end{document}