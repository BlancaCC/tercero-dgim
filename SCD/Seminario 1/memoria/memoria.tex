
% plantilla obtenida de: https://www.overleaf.com/19886281jjqffwsxshmm#/73112823/

\documentclass[a4paper, 11pt]{article}
\usepackage{comment} % enables the use of multi-line comments (\ifx \fi) 
\usepackage{lipsum} %This package just generates Lorem Ipsum filler text. 
\usepackage{fullpage} % changes the margin

\usepackage[spanish]{babel}
\usepackage[utf8]{inputenc}

\usepackage{graphicx}

\newcommand{\imageins}[4]{\begin{figure}[!ht]		%Take the hardwork from using images. Let this command do the work for you. Insert images by just using this command \imageins{filename}{width as a ratio of total text width of the page}{caption name}{label name for referring in articles}		
    \centering
    \includegraphics[width=#2\textwidth]{#1}
    %\caption{#3}
    %\label{#4}
    \vspace{0.2em}
\end{figure}}

%%%%%%%%%%%%%%%%%%%%%%%%%%%%%%%%%%%%%%%%%%%%%%%%%%%%%%%%%%%%%%%%%%%%%%%%%%%%%%%%%%
\usepackage{listings}
\usepackage{color}
 
\definecolor{codegreen}{rgb}{0,0.6,0}
\definecolor{codegray}{rgb}{0.5,0.5,0.5}
\definecolor{codepurple}{rgb}{0.58,0,0.82}
\definecolor{backcolour}{rgb}{0.95,0.95,0.92}
 
\lstdefinestyle{mystyle}{
    backgroundcolor=\color{backcolour},   
    commentstyle=\color{codegreen},
    keywordstyle=\color{magenta},
    numberstyle=\tiny\color{codegray},
    stringstyle=\color{codepurple},
    basicstyle=\footnotesize,
    breakatwhitespace=false,         
    breaklines=true,                 
    captionpos=b,                    
    keepspaces=true,                 
    numbers=left,                    
    numbersep=5pt,                  
    showspaces=false,                
    showstringspaces=false,
    showtabs=false,                  
    tabsize=2
}
 
\lstset{style=mystyle}

%%%%%%%%%%%%%%%%%%%%%%%%%%%%%%%%%%%%%%%%%%%%%%%%%%%%%%%%%%%%%%%%%%%%%%%%%%%%%%%%%%

\begin{document}
%Header-Make sure you update this information!!!!
\noindent
\large\textbf{Cálculo de $\pi$ concurrentemente} \hfill \textbf{Antonio Gámiz Delgado} \\
\normalsize Sistemas Concurrentes y Distribuidos \hfill 21/09/2018
%\normalsize ECE 100-003 \hfill Teammates: Student1, Student2 \\
%Prof. Oruklu \hfill Lab Date: XX/XX/XX \\
%TA: Adam Sumner \hfill Due Date: XX/XX/XX

\section*{Chunks}

\begin{lstlisting}[language=C++, caption=piChunk.cpp]
double funcion_hebra( long i )
{
  double suma = 0.0;
  for( long j = i*chunk_size ; j < (i+1)*chunk_size; j++ )
    if( j < m ) suma += f( (j + double(0.5) )/m );
  return suma;
}

double calcular_integral_concurrente( )
{
  future<double> futuros[n] ;

  // poner en marcha todas las hebras y obtener los futuros
  for( int i = 0 ; i < n ; i++ )
    futuros[i] = async( launch::async, funcion_hebra, i ) ;

  long double pi = 0.0;
  for( int i = 0 ; i < n ; i++ )
     pi += futuros[i].get();

  pi/=m;
  return pi;
}
\end{lstlisting}


\section*{Scatter}

\begin{lstlisting}[language=C++, caption=piScatter.cpp]
double funcion_hebra_scatter( long i )
{
  double suma = 0.0;
  for( long j = i ; j < m; j += n )
    if( j < m ) suma += f( (j + double(0.5) )/m );
  return suma;
}

double calcular_integral_concurrente_scatter( )
{
  future<double> futuros[n] ;

  // poner en marcha todas las hebras y obtener los futuros
  for( int i = 0 ; i < n ; i++ )
    futuros[i] = async( launch::async, funcion_hebra_scatter, i ) ;

  long double pi = 0.0;
  for( int i = 0 ; i < n ; i++ )
     pi += futuros[i].get();

  pi/=m;
  return pi;
}
\end{lstlisting}

\section*{Análisis de los resultados}

A continuación se muestra una tabla con los tiempos de ejecución de los dos programas anteriores, con 1 hebra (secuencial, 2, 4 y 8 respectivamente y repartiendo el trabajo en chunks, y luego en ciclos (scatter).

\imageins{./images/timetable.png}{1}{}{}

Veamos las tablas de las ganancias:

\imageins{./images/chunk_s.png}{0.8}{}{}
\imageins{./images/scatter_s.png}{0.8}{}{}

Como vemos, para $n=2,4,8$ la ganancia es también de 2,4 y 8 respectivamente. Esto también se puede ver en el siguiente gráfico de uso de la CPU para $n=8$, en el que vemos que una de las hebras consume el 100\% de sus recursos durante aproximadamente 8 veces más que el tiempo que están las 8 hebras haciendo el cálculo de forma concurrente:

\imageins{./images/cpu_usage.png}{1}{}{}

\newpage

Aquí podemos ver la ganancia de cada uno de los programas con respecto a la versión secuencial: 

\imageins{./images/chunk_g.png}{0.8}{}{}
\imageins{./images/scatter_g.png}{0.8}{}{}

\end{document}
