
% plantilla obtenida de: https://www.overleaf.com/19886281jjqffwsxshmm#/73112823/

\documentclass[a4paper, 11pt]{article}
\usepackage{comment} % enables the use of multi-line comments (\ifx \fi) 
\usepackage{lipsum} %This package just generates Lorem Ipsum filler text. 
\usepackage{fullpage} % changes the margin

\usepackage[spanish]{babel}
\usepackage[utf8]{inputenc}
\decimalpoint
\usepackage{graphicx}

\usepackage{amsmath}
\usepackage{amsfonts}
% or
\usepackage{amssymb}
\usepackage{tikz}
\usepackage{array}
\newcolumntype{C}{>{$}l<{$}} % math-mode version of "l" column type
\newcommand{\imageins}[4]{\begin{figure}[!ht]		%Take the hardwork from using images. Let this command do the work for you. Insert images by just using this command \imageins{filename}{width as a ratio of total text width of the page}{caption name}{label name for referring in articles}		
    \centering
    \includegraphics[width=#2\textwidth]{#1}
    %\caption{#3}
    %\label{#4}
    \vspace{0.2em}
\end{figure}}

%%%%%%%%%%%%%%%%%%%%%%%%%%%%%%%%%%%%%%%%%%%%%%%%%%%%%%%%%%%%%%%%%%%%%%%%%%%%%%%%%%
\usepackage{listings}
\usepackage{color}
\newcommand{\R}[1][]{\mathbb{R}^{#1}}
\newcommand{\N}[1][]{\mathbb{N}^{#1}}
\newcommand{\solution}[1]{\text{\fbox{$#1$}}}

\DeclareMathOperator\arctanh{arctanh}

\newcommand{\abs}[1]{\left|#1\right|}

\newcommand{\FVCP}[5]{x(t)=e^{#1}\int_{#4}^{t}e^{#2}#3ds+#5e^{#1}}
\newcommand{\FVC}{x(t)=e^{\int_{t_0}^{t}a(s)ds}\int_{t_0}^te^{-\int_{t_0}^{s}a(z)dz}b(s)ds+x_0e^{\int_{t_0}^{t}a(s)ds}, \quad x(t_0)=x_0}


\newcommand{\tick}{\textbf{\color{green}{ (\checkmark) }}}
\newcommand{\warning}{\textbf{\color{red}{ {\fontencoding{U}\fontfamily{futs}\selectfont\char 66\relax} }}}

\newcommand{\qued}{\hfill$\blacksquare$}

\newcommand{\p}[2]{\frac{\partial#1}{\partial#2}}


%%%%%%%%%%%%%%%%%%%%%%%%%%%%%%%%%%%%%%%%%%%%%%%%

\newenvironment{aclaration}    
{
\begin{center}
\begin{tabular}{|p{0.9\textwidth}|}
\hline \\ \warning
}{
\\\hline
\end{tabular} 
\end{center}
}

\definecolor{codegreen}{rgb}{0,0.6,0}
\definecolor{codegray}{rgb}{0.5,0.5,0.5}
\definecolor{codepurple}{rgb}{0.58,0,0.82}
\definecolor{backcolour}{rgb}{0.95,0.95,0.92}
 
\lstdefinestyle{mystyle}{
    backgroundcolor=\color{backcolour},   
    commentstyle=\color{codegreen},
    keywordstyle=\color{magenta},
    numberstyle=\tiny\color{codegray},
    stringstyle=\color{codepurple},
    basicstyle=\footnotesize,
    breakatwhitespace=false,         
    breaklines=true,                 
    captionpos=b,                    
    keepspaces=true,                 
    numbers=left,                    
    numbersep=5pt,                  
    showspaces=false,                
    showstringspaces=false,
    showtabs=false,                  
    tabsize=2
}
 
\lstset{style=mystyle}

%%%%%%%%%%%%%%%%%%%%%%%%%%%%%%%%%%%%%%%%%%%%%%%%%%%%%%%%%%%%%%%%%%%%%%%%%%%%%%%%%%

\begin{document}
%Header-Make sure you update this information!!!!
\noindent
\large\textbf{Parcial I} \hfill \textbf{Antonio Gámiz Delgado} \\
\normalsize Probabilidad \hfill 28/11/2018
%\normalsize ECE 100-003 \hfill Teammates: Student1, Student2 \\
%Prof. Oruklu \hfill Lab Date: XX/XX/XX \\
%TA: Adam Sumner \hfill Due Date: XX/XX/XX

\subsubsection*{Cálculo de la función de distribución de un vector aleatorio continuo.}

Sea $f(x,y)$ la función de densidad de dos variables aleatorias continuas $(X,Y)$, entonces:
\[
F_X(x_0,y_0)=P[X\leq x_0, Y\leq y_0] = \int_{(-\infty, x_0),(-\infty,y_0)}f(x,y)dxdy=\int_{-\infty}^{x_0}\int_{-\infty}^{y_0}f(x,y)dxdy
\]

\subsubsection*{Cálculo de probabilidades usando la función de densisdad.}

Dado un vector $(X,Y)$ con función de densidad $f(x,y)$, la probabilidad de que dicho vector pertenezca a un conjunto de Borel $B$ se calcula como:
\[
\int_Bf(x,y)dxdy \quad \forall B \in \R[2]
\]

Si $f(x,y)$ es no nula sólo en un conjunto de valores $C$, la expresión se reduce a:

\[
\int_{B\cap C}f(x,y)dxdy
\]

\subsubsection*{Distribuciones marginales.}
\begin{itemize}
\item \underline{Vectores discretos:}  Sea $X=(X_1,X_2)$ un vector aleatorio discreto en $\R[2]$ con función masa de probabilidad:
\[
\{\left((x_{1i},x_{2j}), \enskip i=1,\dotso, \infty, \enskip j=1,\dotso, \infty\right)\}
\]

Sus funciones masa de probabilidad marginales serán:
\[
P[X_1=x_{1i}]=P[X_1=x_{1i},x_2\in\R]=\sum_{j}^\infty p_{ij}=p_{i\cdot} \enskip \forall i=1,\dotso, \infty
\]
\[
P[X_2=x_{2j}]=P[x_1\in\R, X_2=x_{2j},]=\sum_{i}^\infty p_{ij}=p_{\cdot j} \enskip \forall j=1,\dotso, \infty
\]

Verificando que $\sum_i p_{i\cdot}=\sum_j p_{\cdot j } = 1.$

Y las funciones de distribución marginales serán:
\[
F_1(x_1)=P[X_1\leq x_1, x_2\in\R]=F(x_1,+\infty)=\sum_{x_{1i}\leq x_1}p_{i\cdot}=\sum_{x_{1i}\leq x_1}\sum_{j=1}^\infty p_{ij}
\]
\[
F_2(x_2)=P[x_1\in\R, X_2\leq x_2]=F(+\infty,x_2)=\sum_{x_{2j}\leq x_2}p_{\cdot j}=\sum_{x_{2j}\leq x_2}\sum_{i=1}^\infty p_{ij}
\]

\item \underline{Vectores continuos:}  Sea $X=(X_1,X_2)$ un vector aleatorio continuo en $\R[2]$ con función de densidad $f(x_1,x_2)$. Entonces las marginales son:
\[
f_1(x_1)=\int_{-\infty}^{+\infty}f(x_1,x_2)dx_2 \quad f_2(x_2)=\int_{-\infty}^{+\infty}f(x_1,x_2)dx_1
\]

\[
F(x_1)=P[X_1\leq x_1, X_2\in\R]=\int_{-\infty}^{x_1}f_1(s_1)ds_1=\int_{-\infty}^{x_1}\int_{-\infty}^{+\infty}f(s_1, s_2)ds_2ds_1
\]
\[
F(x_2)=P[X_1\in\R, X_2\leq x_2]=\int_{-\infty}^{x_2}f_2(s_2)ds_2=\int_{-\infty}^{x_2}\int_{-\infty}^{+\infty}f(s_1, s_2)ds_1ds_2\]

\end{itemize}

\subsubsection*{Distribuciones condicionadas.}

\begin{itemize}
\item \underline{Vectores discreto:} Sea $X=(X_1,X_2)$ un vector aleatorio discreto en $\R[2]$ con función masa de probabilidad:
\[
\{\left((x_{1i},x_{2j}), \enskip i=1,\dotso, \infty, \enskip j=1,\dotso, \infty\right)\}
\]

Entonces para cada valor $X_2=x_{2j}$ es posible obtener la distribución condicionada de $X_1$ a ese valor, denotándose $x_1|X_2=x_{2j}$. Su función masa de probabilidad condicionada es:
\[
P[X_1=x_{1i}|X_2=x_{2j}]=\frac{p_{ij}}{p_{\cdot j}} \enskip \forall p_{\cdot j} \neq 0 \quad P[X_2=x_{2j}|X_1=x_{1i}]=\frac{p_{ij}}{p_{i\cdot}} \enskip \forall p_{i\cdot} \neq 0
\]
\[
F(x_1|X_2=x_{2j})=\sum_{x_{1i}\leq x_1}p_{i/j}=\frac{\sum_{x_{1i}\leq x_1}p_{ij}}{p_{\cdot j}} \enskip \forall x_1\in\R
\]
\[
F(x_2|X_1=x_{1i})=\sum_{x_{2j}\leq x_2}p_{i/j}=\frac{\sum_{x_{2j}\leq x_2}p_{ij}}{p_{i\cdot}} \enskip \forall x_2\in\R
\]

\item \underline{Vectores continuos:}  Sea $X=(X_1,X_2)$ un vector aleatorio continuo en $\R[2]$ con función de densidad $f(x_1,x_2)$. Entonces las condicionadas son:
\[
f(x_1|X_2=x_2)=\frac{f(x_1,x_2)}{f_2(x_2)} \quad f(X_1=x_1|x_2)=\frac{f(x_1,x_2)}{f_1(x_1)}
\]
\[
F(x_1|X_2=x_2)=P(X_1\leq x_1,X_2=x_2)=\int_{-\infty}^{x_1}f(s_1|X_2=x_2)ds_1=\frac{\displaystyle\int_{-\infty}^{x_1}f(s_1,x_2)ds_1}{f_2(x_2)}
\]
\[
F(x_2|X_1=x_1)=P(X_1=x_1,X_2\leq x_2)=\int_{-\infty}^{x_2}f(X_1=x_1|s_2)ds_2=\frac{\displaystyle\int_{-\infty}^{x_2}f(x_1,s_2)ds_2}{f_1(x_1)}
\]

\end{itemize}

\subsubsection*{Cambio de variable.}
\begin{itemize}
\item \underline{Cambio de discreto a discreto:}
\[
P(Y=y)=P(g(x)=y)=P(x\in g^{-1}(y))=\sum_{g^{-1}(y)}P(X=x) \quad \forall y\in g(E_X), \enskip E_X\in \R[n]
\]
\begin{itemize}
\item Ejemplo: Calcular la función masa de probabilidad de $Y=(|x_1|, x_2^2)$:
\[\begin{array}{|c|c|c|c|}
\hline
X_2/X_1 & -1 & 0 & 1 \\ 
\hline
-2 & \frac{1}{6} & \frac{1}{12} & \frac{1}{6} \\ 
\hline
1 & \frac{1}{6} & \frac{1}{12} & \frac{1}{6} \\ 
\hline
2 & \frac{1}{12} & 0 & \frac{1}{12} \\
\hline
\end{array} \Longrightarrow \begin{array}{|c|c|c|}
\hline
Y_2/Y_1 & 0 & 1 \\
\hline
1 & \frac{1}{12} & \frac{1}{3} \\
\hline
4 & \frac{1}{12} & \frac{1}{2} \\
\hline
\end{array}
\]

\end{itemize}
\item \underline{Cambio de continuo a discreto:}
\[
P(Y=y)=P(g(x)=y)=P(x\in g^{-1}(y))=\int_{g^{-1}(y)}f_X(x)dx
\]

\begin{itemize}
\item Ejemplo: Sea $X=(X_1, X_2)$ y $f(x_1,x_2)=\lambda\mu e^{-\lambda x_1}e^{-\mu x_2}, \enskip x_1,x_2,\lambda,\mu>0$.
\[
Y = \left\{\begin{array}{cc}
0 & x_1 > x_2\\
1 & x_1 < x_2
\end{array}
\right.
\]

\[
P[Y=0]=\int\int f_X(x)dx=\int_0^\infty\int_{x_2}^\infty f(x_1,x_2)dx_1dx_2 = \frac{\mu}{\lambda+\mu}
\]
\[
P[Y=1]=1-P[Y=0] = \frac{\lambda}{\lambda+\mu}
\]
\end{itemize}

\item \underline{Cambio de continuo a continuo:}
\[
F_y(y)=P(x\in g^{-1}((-\infty,y])=\int_{g^{-1}((-\infty,y])}f_X(x)dx
\]

\begin{itemize}
\item Ejemplo: $X=(x,y)$, $f(x,y)=e^{-x}e^{-y}$, $x,y>0$. $U=\frac{x}{x+y}$.

Vemos que $0<u<1$.
\[
P(U\leq u) = P\left(\frac{x}{x+y}\leq u\right)=P\left(x\leq \frac{u}{1-u}y\right)=\int_0^\infty\int_0^{\frac{u}{1-u}y}e^{-x}e^{-y}dxdy=u.
\]
Por lo que su función de densidad será:
\[
f_U(u)=\frac{\partial F_U(u)}{\partial u } = 1.
\]
\end{itemize}
\end{itemize}

\subsubsection*{Teorema: Cambio de variable de continuo a continuo.}
Sea $X$ un vector aleatorio continuo $n$-dimensional y $g:(\R[n],B^n)\longrightarrow (\R[n],B^n)$, con función de densidad $f_X$, un transformación medible tal que:
\begin{itemize}
\item $g(y_1,\dotso,y_n)$ admite inverse $g^{-1}(y_1^*,\dotso, y_n^*)$
\item $g^{-1}$ sea derivable en todos los argumentos:
\[
\forall i,j=1,\dotso, n \enskip \displaystyle\exists\displaystyle\frac{\partial g_i^*(y1,\dotso,y_n)}{\partial y_j}
\]
\item El jacobiano de $g^{-1}$ sea no nulo:
\[
\left|J\right|=\left|\left(\frac{\partial g_i^*(y_1,\dotso,y_n)}{\partial y_j}\right)_{ij}\right|\neq 0
\]
\end{itemize}
Entonces $Y=g(X)$ es un vector aleatorio continuo con función de densidad:
\[
f_Y(y)=f_x(g^{-1}(y))\cdot |J| \enskip \forall y\in\R
\]
\begin{itemize}
\item \underline{Nota 1:} Si el vector transformado es de dimensión inferior al original, se toman variables auxiliares y se calcula la marginal correspondiente.
\item \underline{Nota 2:} Si g no admite inversa pero cada $g_i$ tiene un número finito de antimágnes ($g_1^*(y),\dotso, g_k^*(y)$) cumpliendo las condiciones del teorema, entonces:
\[
f_Y(y)=\sum_{i=0}^k f_X(g_i^*(y))\cdot|J_i|
\]

\begin{itemize}
\item Ejemplo: $X=(x,y)$, $f(x,y)=e^{-x}e^{-y}$, $x,y>0$. Cambio $u=\frac{x}{x+y}$. 

Necesitamos una variable auxiliar $v=x+y$. Nuestra $g$ será:
\[
g(x_1,x_2)=(g_1(x_1,x_2), g_2(x_1,x_2))=\left(\frac{x}{x+y}, x+y\right)
\]

Calculamos la transformada inversa:
\[
g^{-1}(u,v)=(uv,v(1-u)),\quad 0<u<1,v>0.
\]

Vemos que esa transformada inversa cumple las dos primeras condiciones. Veámos si se cumple la tercera:
\[
|J|=\left|\begin{array}{cc}
v & u\\
-v & 1-u
\end{array}
\right| = v \neq 0
\]

Luego se cumplen las 3 condiciones del teorema:
\[f_U(u,v) = f_X(u,v)=e^{-uv}e^{-v(1-u)}\cdot v = ve^{-v}
\]

Como lo que queremos es la función de densidad de $u$, calculamos su marginal:
\[
f_v(u)=\int_0^\infty ve^{-v}dv=1, \quad 0<u<1.
\]

\end{itemize}

\end{itemize}

\subsubsection*{Distribución del máximo y del mínimo.}

\end{document}