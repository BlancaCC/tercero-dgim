\documentclass[11pt]{article}
% decent example of doing mathematics and proofs in LaTeX.
% An Incredible degree of information can be found at
% http://en.wikibooks.org/wiki/LaTeX/Mathematics

% Use wide margins, but not quite so wide as fullpage.sty

\usepackage[spanish]{babel}
\usepackage[utf8]{inputenc}
\marginparwidth 0.5in 
\oddsidemargin 0.25in 
\evensidemargin 0.25in 
\marginparsep 0.25in
\topmargin 0.25in 
\textwidth 6in \textheight 8 in
% That's about enough definitions

\usepackage{amsmath}
\usepackage{upgreek}

\begin{document}
\author{Antonio Gámiz Delgado}
\title{Relación Preliminar: Probabilidad}
\maketitle

\begin{enumerate}

\item 

Se estudian las plantas de una determinada zona donde ha atacado un virus. La probabilidad de que cada planta este contaminada es 0.35.

Sea $X= \text{ número de plantas contaminadas en } n \text{ análisis con una probabilidad } p=0.35.  \Longrightarrow X \sim B(n,0.35)$ 

\begin{enumerate}
\item ¿Cuál es el número esperado de plantas contaminadas en 5 analizadas?

Necesitamos calcular la esperanza de esa distribución binomial para $n=5$.
\[
E[X]=np=5*0.35=1.75
\]
Luego se esperan 2 plantas contaminadas en 5 muestras.

\item Calcular la probabilidad de encontrar entre 2 y 5 plantas contaminadas en 9 exámenes.

\[
P(2 \leq X \leq 5) = \sum_{i=2}^5P(X=i) = \sum_{i=2}^5\binom{9}{i}p^i(1-p)^{9-i}
\]
\[
= \binom{9}{2}0.35^2*0.65^7+\binom{9}{3}0.35^3*0.65^6+\binom{9}{4}0.35^4*0.65^5+\binom{9}{5}0.35^5*0.65^4= 
\]
\[
= 0.825326770734375
\]

\item Hallar la probabilidad de encontrar 4 plantas no contaminadas en 6 análisis.

\[
P(X=4)=\binom{9}{4}0.35^40.65^5=0.21938630101171875
\]
\end{enumerate}

\item 

Cada que vez que una máquina, dedicada a la fabricación de comprimidos, produce uno, la probabilidad de que sea defectuoso es 0.01.

\begin{enumerate}
\item Si los comprimidos se colocan en tubos de 25, ¿cuál es la probabilidad de que en un tubo todos los comprimidos sean buenos?

Sea $X = \{\text{ número de comprimidos defectuosos } \} \Longrightarrow X \sim B(n,0.01)$
\[
P(X=0)=\binom{25}{0}0.01^00.99{25} = 0.777821359
\]

\item Si los tubos se colocan en cajas de 10, ¿cuál es la probabilidad de que en una determinada caja haya exactamente 5 tubos con un comprimido defectuoso?

Calculamos la probabilidad de que un tubo tenga un comprimido defectuoso:

\[
P(X=1)=\binom{25}{1}0.01*0.99^{24} =0.1964
\]

Podemos hacer $X \sim B(10, 0.1964)$ con $X = \{\text{número de tubos con un comprimido defectuoso}\} $, ya que el tamaño de la caja es 10 tubos y sólo estamos calculando la probabilidad para una caja. Luego:

\[
P(X=5)=\binom{10}{5}0.1964^5(1-0.1964)^5=0.02468714
\]
\end{enumerate}

\item 

Un pescador desea capturar un ejemplar de sardina que se encuentra siempre en una determinada zona del mar con probabilidad 0.15. Hallar la probabilidad de que tenga que pescar 10 peces de especies distintas de la deseada antes de:

Sigue una distribución binomial negativa con probabilidad de éxito $p=0.15 \Longrightarrow X \sim BN(r,0.15)$, siendo $X=\{$número de peces de diferente especie pescados antes de haber pescado $k$ sardinas.$\}$

\begin{enumerate}

\item pescar la sardina buscada.

\[
P(X="\text{10 antes de pescar 1 sardinas}")=BN(1,0.15)=
\]
\[
=\binom{x+r-1}{x}(1-p)^xp^r=\binom{10}{10}(1-0.15)^{10}0.15=0.02953
\]
\item pescar tres ejemplares de la sardina buscada.

\[
P(X="\text{10 antes de pescar 3 sardinas}")=BN(3,0.15)=
\]
\[
=\binom{x+r-1}{x}(1-p)^xp^r=\binom{12}{10}(1-0.15)^{10}0.15^3=0.04385
\]

\end{enumerate}

\item Un científico necesita 5 monos afectados por cierta enfermedad para realizar un experimento. La incidencia de la enfermedad en la población de monos es siempre del 30\%. El científico examinará uno a uno los monos de un gran colectivo, hasta encontrar 5 afectados por la enfermedad.

Sigue una distribución binomial negativa con probabilidad de éxito $p=\frac{3}{10} \Longrightarrow X \sim BN(5,\frac{3}{10})$, siendo $X=\{$número de monos sanos encontrados antes de haber encontrado $5$ monos afectados.$\}$.


\begin{enumerate}
\item Calcular el número medio de exámenes requeridos.

\[
E[X]=\frac{r(1-p)}{p}=\frac{5(1-\frac{3}{10})}{\frac{3}{10}}=11.666667
\]

Así que, de media, el científico necesitará examinar a 12 monos para encontrar 5 afectados.

\item Calcular la probabilidad de que encuentre 10 monos sanos antes de encontrar los 5 afectados.

\[
P(X=10)=\binom{x+r-1}{x}(1-p)^xp^r=\binom{14}{10}(1-0.33)^{10}*0.33^5=0.06871
\]

\item Calcular la probabilidad de que tenga que examinar por lo menos 20 monos.

\[
P(X=20)=\binom{x+r-1}{x}(1-p)^xp^r=\binom{24}{20}(1-0.33)^{20}*0.33^5=0.0206
\]


\end{enumerate}

\item Se capturan 100 peces de un estanque que contiene 10000. Se les marca con una anilla y se devuelven al agua. Transcurridos unos días se capturan de nuevo 100 peces y se cuentan los anillados.

Sigue una distribución hipergeométrica  $X \sim H(10000, 100, 100)$, siendo $X=\{$número de peces anillados encontrados en la segunda captura.$\}$.

\begin{enumerate}

\item Calcular la probabilidad de que en la segunda captura se encuentre al menos un pez anillado.

\[
P(X=1)=\frac{\binom{N_1}{x}\binom{N-N_1}{n-x}}{\binom{N}{n}}=\frac{\binom{100}{1}\binom{9900}{99}}{\binom{10000}{100}}=0.3715891
\]

\item Calcular el número esperado de peces anillados en la segunda captura.

La probabilidad de sacar un pez anillado es $p=\frac{N_1}{N}=\frac{100}{10000}=0.01$

\[
E[X]=np=100*0.01=1
\]

Así que se espera obtener 1 pez en la segunda captura.

\end{enumerate}

\item Cada página impresa de un libro tiene 40 líneas, y cada línea tiene 75 posiciones. Se supone que la probabilidad de que en cada posición haya un error es 1/6000.

Sigue una distribución binomial con $n=40*75=3000$ y $p=1/6000 \Longrightarrow X \sim B(3000, 1/6000$, siendo $X=\{$número de posiciones de impresión erróneas encontradas en una página$\}$.


\begin{enumerate}
\item ¿Cuál es la distribución del número de errores por página?
\[
Var[X] = np(1-p) = 3000\frac{1}{6000}(1-\frac{1}{6000})=0.4999 \Longrightarrow \sigma [X] = \sqrt{Var[X]} = 0.70704
\]

\item Calcular la probabilidad de que una página no contenga errores y de que contenga como mínimo 5 errores.

\[
P(X=0)= (\frac{1}{6000})^{3000} = 0.6065
\]
\[
P(X \geq 5)=1 - P(X<=4) = 1 - \sum_{i=0}^4\binom{3000}{i}(\frac{1}{6000})^i(1-\frac{1}{6000})^{3000-i}=1-0.9998 = 0.0002
\]

\item ¿Cuál es la probabilidad de que un capítulo de 20 páginas no contenga errores?

\[
X \sim B(20, 0.6065)
\]
\[
P(X=0) = (1-0.6065)^{20} = 7.92309 * 10^{-9}
\]

\end{enumerate}

\item En un departamento de control de calidad se inspeccionan las unidades terminadas que provienen de una línea de ensamble. La probabilidad de que cada unidad sea defectuosa es 0.05.

Sigue una distribución binomial negativa con $p=0.05 \Longrightarrow X \sim BN(r, 0.05$, siendo $X=\{$número de unidades defectuosas encontradas$\}$.

\begin{enumerate}
\item ¿Cúal es la probabilidad de que la vigésima unidad inspeccionada sea la segunda que se encuentra defectuosa?
\[
X \sim BN(2, 0.05)
\]
\[
P(X=20) = \binom{x+r-1}{x}(1-p)^xp^r=\binom{21}{20}(1-0.05)^{20}0.05^2=0.0188
\]

\item ¿Cuántas unidades deben inspeccionarse por término medio hasta encontrar 4 defectuosas?

\[
E[X] = \frac{r(1-p)}{p}=\frac{4(1-0.05)}{0.05} = 76
\]

\item Calcular la desviación típica del número de unidades inspeccionadas hasta encontrar 4 defectuosas.
\[
Var[X] = \frac{r(1-p)}{p^2}=\frac{4(1-0.05)}{0.05^2}=1520 \Longrightarrow \sigma [X] = \sqrt{Var[X]} = 38.98717
\]

\end{enumerate}

\item Los números 1,2,3,...,10 se escriben en 10 tarjetas y se colocan en una urna. Las tarjetas se extraen una a una y sin devolución. Calcular las probabilidades de los siguientes sucesos:

\begin{enumerate}
\item Hay exactamente tres números pares en cinco extracciones.
\[
X \sim H(10,5,5)
\]
\[
P(X=3)=\frac{\binom{N_1}{x}\binom{N-N_1}{n-x}}{\binom{N}{n}}=\frac{\binom{5}{3}\binom{5}{2}}{\binom{10}{5}}=0.396825
\]
\item Se necesiten 5 extracciones para obterner 3 números pares.

\[
X \sim H(10, 5, 2)
\]
\[
P(X=2)=\frac{\binom{N_1}{x}\binom{N-N_1}{n-x}}{\binom{N}{n}}=\frac{\binom{5}{2}\binom{5}{2}}{\binom{10}{4}}=0.476109
\]

Ahora tenemos la probabilidad de haber sacado 4 papeletas, en la que hemos obtenido 2 impares y 2 pares. 

Como son sucesos independientes, podemos hacer:
\[
P(A\cap B)=P(A)P(B)=0.5*0.476109=0.238095
\]
Con A la probabilidad de sacar un número par y B la probabilidad de haber sacado 2 pares e impares en 4 extracciones.

\item Obterner el número 7 en la cuarta extracción.

Si obtenemos el número 7 en la cuarta extracción ($X$), significa que antes hemos sacado 3 números distintos de 7: 
\[
P(X)=(1-\frac{1}{10})(1-\frac{1}{9})(1-\frac{1}{8})\frac{1}{7} = 0.1
\]
\end{enumerate}

\item Supongamos que el número de televisores vendidos en un comercio durante un mes se distribuye según una Poisson de parámetro 10, y que el beneficio neto por unidad es 30 euros.
\[
X \sim \mathcal{P}(10)
\]
\begin{enumerate}
\item ¿Cuál es la probabilidad de que el beneficio neto obtenido por un comerciante durante un mes sea al menos de 360 euros?

Si queremos que el beneficia sea de 360 euros, entonces debemos vender 12 televisores, luego:
\[
P(X=12)=1-\sum_{i=0}^{11}e^{-\lambda}\frac{\lambda^i}{i!}=0.30322
\]

\item ¿Cuántos televisores debe tener el comerciante a principio de mes para tener al menos proabilidad de 0.95 de satisfacer toda la demanda?
\end{enumerate}

\end{enumerate}
\end{document}
