\documentclass[11pt,a4paper]{article}

% Packages
\usepackage[utf8]{inputenc}
\usepackage[spanish, es-tabla]{babel}
\usepackage{caption}
\usepackage{listings}
\usepackage{adjustbox}
\usepackage{enumitem}
\usepackage{boldline}
\usepackage{amssymb, amsmath,amsthm}
\usepackage[margin=1in]{geometry}
\usepackage{xcolor}
\usepackage{soul}
\usepackage{upgreek}
\usepackage{float}

% Meta
\title{Entrega 5}
\date{}
% Custom
\providecommand{\abs}[1]{\lvert#1\rvert}
\setlength\parindent{0pt}
% Redefinir letra griega épsilon.
\let\epsilon\upvarepsilon
% Fracciones grandes
\newcommand\ddfrac[2]{\frac{\displaystyle #1}{\displaystyle #2}}
% Primera derivada parcial: \pder[f]{x}
\newcommand{\pder}[2][]{\frac{\partial#1}{\partial#2}}

\begin{document}

\maketitle

\section*{Ejercicio 13}
Encuentra una gramática libre de contexto en forma normal de Chomsky que genere el siguiente lenguaje:

$$ L = \{ucv : u, v \in \{0, 1\} \text{ y nº de subcadenas } 01 \text{ en } u \text{ es igual al nº de subcadenas } 10 \text{ en } v\} $$

Comprueba con el algoritmo \textbf{CYK} si la cadena \emph{010c101} pertenece al lenguaje generado por la gramática. \\

\begin {enumerate} 

\item Gramática libre del contexto que genera el lenguaje:
	
$$ S \rightarrow S_1AS_1 | S_1BS_0 | S_0CS_1 | S_0DS_0 | YXY $$
$$ Y \rightarrow S_1 | S_0 | \epsilon $$
$$ S_0 \rightarrow 0 | 0S_0$$
$$ S_1 \rightarrow 1 | 1S_1$$
$$ A \rightarrow 1A | A1 | B0 | 0C | X | c $$
$$ B \rightarrow 1B | B0 | 0D | X | c $$
$$ C \rightarrow 0C | C1 | C0 | X | c $$
$$ D \rightarrow 0D | D0 | X | c $$
$$ X \rightarrow 01A10 $$


\item Gramática en forma normal de Chomsky que genere el lenguaje:

$$ S \rightarrow S_1E_2 | S_0E_3 | YX | XY | S_0X_0$$
$$ E_2 \rightarrow AS_1 | BS_0 $$
$$ E_3 \rightarrow CS_1 | DS_0 $$
$$ S_0 \rightarrow 0 | S_0S_0$$
$$ S_1 \rightarrow 1 | S_1S_1$$
$$ A \rightarrow S_1A | AS_1 | BS_0 | S_0C | S_0X_0 | c $$
$$ B \rightarrow S_1B | BS_0 | S_0D | S_0X_0 | c $$
$$ C \rightarrow S_0C | CS_1 | DS_0 | S_0X_0 | c $$
$$ D \rightarrow S_0D | DS_0 | S_0X_0 | c $$
$$ X \rightarrow S_0X_0$$
$$ X_0 \rightarrow S_1X_1$$
$$ X_1 \rightarrow AX_2$$
$$ X_2 \rightarrow S_1S_0$$

\item Apliquemos ahora \textbf{CYK} para la cadena \emph{010c010}:

\begin{table}[h]
	\centering
	\caption{\textbf{CYK} - ejercicio 13}
	\label{my-label}
	\begin{tabular}{lllllll}
		0                           & 1                                & 0                                        & c                                      & 0                          & 1                          & 0                          \\ \hline
		\multicolumn{1}{|l|}{$S_0$} & \multicolumn{1}{l|}{$S_1$}       & \multicolumn{1}{l|}{$S_0$}               & \multicolumn{1}{l|}{$A,B,C,D$}         & \multicolumn{1}{l|}{$S_0$} & \multicolumn{1}{l|}{$S_1$} & \multicolumn{1}{l|}{$S_0$} \\ \hline
		\multicolumn{1}{|l|}{}      & \multicolumn{1}{l|}{$X_2$}       & \multicolumn{1}{l|}{$A,B,C,D$}           & \multicolumn{1}{l|}{$A,B,C,D,E_2,E_3$} & \multicolumn{1}{l|}{}      & \multicolumn{1}{l|}{$X_2$} &                            \\ \cline{1-6}
		\multicolumn{1}{|l|}{}      & \multicolumn{1}{l|}{$A,B$}       & \multicolumn{1}{l|}{$S,A,B,C,D,E_2,E_3$} & \multicolumn{1}{l|}{$A,C,E_2,E_3$}     & \multicolumn{1}{l|}{}      &                            &                            \\ \cline{1-5}
		\multicolumn{1}{|l|}{}      & \multicolumn{1}{l|}{$S,A,B,E_2$} & \multicolumn{1}{l|}{$S,A,C,E_2,E_3$}     & \multicolumn{1}{l|}{$X_1$}             &                            &                            &                            \\ \cline{1-4}
		\multicolumn{1}{|l|}{}      & \multicolumn{1}{l|}{$S,A$}       & \multicolumn{1}{l|}{$X_1$}               &                                        &                            &                            &                            \\ \cline{1-3}
		\multicolumn{1}{|l|}{}      & \multicolumn{1}{l|}{$X_0,X_1$}   &                                          &                                        &                            &                            &                            \\ \cline{1-2}
		\multicolumn{1}{|l|}{$S$}   &                                  &                                          &                                        &                            &                            &                            \\ \cline{1-1}
	\end{tabular}
\end{table}

\end{enumerate}


\section*{Ejercicio 15}
Encuentra una gramática libre de contexto en forma normal de Chomsky que genere los siguientes lenguaje definidos sobre el alfabeto $\{a, 0, 1\}$ :
$$L_1 = \{auava : u,v \in \{0, 1\}^+ \text { y } u^{-1} = v\} $$
$$L_2 = \{uvu : u,v \in \{0, 1\}^+ \text{ y } u^{-1} = v\} $$
Comprueba con el algoritmo \textbf{CYK} si la cadena \emph{a0a0a} pertenece a $L_1$ y la cadena \emph{011001} pertenece al lenguaje $L_2$ . \\

\begin{enumerate}
	\item Gramática libre del contexto que genera $L_1$:
		
	$$ S \rightarrow aAa $$
	$$ A \rightarrow XAX | a $$
	$$ X \rightarrow 0|1 $$
		
	\item Gramática en forma normal de Chomsky que genera $L_1$:
	
	$$ S \rightarrow AE_1 $$
	$$ E_1 \rightarrow BA $$
	$$ A \rightarrow a $$
	$$ B \rightarrow XE_2 | a $$
	$$ E_2 \rightarrow BX $$
	$$ X \rightarrow 0|1 $$
	
	\item Apliquemos ahora \textbf{CYK} para la cadena \emph{a0a0a}:
	
	\begin{table}[h]
		\centering
		\caption{\textbf{CYK} - ejercicio 15}
		\label{my-label}
		\begin{tabular}{lllll}
			a                            & 0                          & a                           & 0                        & a                           \\ \hline
			\multicolumn{1}{|l|}{$A, B$} & \multicolumn{1}{l|}{$X$}   & \multicolumn{1}{l|}{$A, B$} & \multicolumn{1}{l|}{$X$} & \multicolumn{1}{l|}{$A, B$} \\ \hline
			\multicolumn{1}{|l|}{$E_2$}  & \multicolumn{1}{l|}{}      & \multicolumn{1}{l|}{$E_2$}  & \multicolumn{1}{l|}{}    &                             \\ \cline{1-4}
			\multicolumn{1}{|l|}{}       & \multicolumn{1}{l|}{$B$}   & \multicolumn{1}{l|}{}       &                          &                             \\ \cline{1-3}
			\multicolumn{1}{|l|}{}       & \multicolumn{1}{l|}{$E_1$} &                             &                          &                             \\ \cline{1-2}
			\multicolumn{1}{|l|}{$S$}    &                            &                             &                          &                             \\ \cline{1-1}
		\end{tabular}
	\end{table}
\end{enumerate}


\section*{Ejercicio 21}
Si $L_1$ y $L_2$ son lenguajes sobre el alfabeto $A$, entonces se define el cociente: $$L_1/L_2 = \{u \in A^* : \exists w \in L_2 \text { tal que } uw \in L_1 \}$$ Demostrar que si $L_1$ es independiente del contexto y $L_2$ regular, entonces $L_1/L_2$ es independiente del contexto. \\

\end{document}